\chapter{Introduction}

The rest of this template is just text from a final report submitted to the Air Force Office of Sponsored Research (AFOSR) found at \url{https://apps.dtic.mil/sti/pdfs/AD1104381.pdf}. This is given to provide students some practical examples of how latex works. This is to help ERAU students only. Please do not post it in any forum public or otherwise. \textbf{If a github version of this is maintained please ensure that this chapter and following chapters are deleted.}

It has recently been observed that the large-scale motions within turbulent boundary layers interact with the smaller scales in the flow, in a non-linear manner, through a process of amplitude and frequency modulation~\citep{Mathis2009a,Mathis2009b,Jacobi2013a,Smits2011a,Marusic2010a,Marusic2010b}. Here the term large-scale refers to flow structures larger than the some outer length scale such as the boundary layer thickness. The long term goal of this work is to control wall turbulence by exploiting this non-linear interaction through large-scale inputs to the flow. The large-scale motions within wall turbulence are attractive targets for active flow control as the scale size of devices and sensors required are much more feasible from an engineering perspective. The required frequency response of such sensors and actuators are fairly low (a few 100 Hz). The larger scales are also the dominant energy carrying eddies at high Reynolds numbers and persist for long distances (or time periods), thereby increasing the effective region under control~\citep{Smits2011a}. These practical considerations along with the recent advances in understanding the interactions between the large-scale and small-scales within a boundary layer, makes targeting the large-scales of wall turbulence an attractive proposition. 

\section{This is a Section Heading}
The modulation effect of the large-scale on the smaller scales were initially observed by \citet{Brown1977a} and then by \citet{Bandyopadhyay1984a}. However, after receiving very little attention for almost two decades, several recent studies have focused on the interaction between the large-scale and small-scale motions within a boundary layer, particularly with increasing Reynolds number~\citep{Mathis2009a,Mathis2009b,Chung2010a,Marusic2010a,Guala2011a,Bernardini2011a,Ganapathisubramani2012a,Jacobi2013a,Harun2013a,Agostini2014a,Talluru2014b,Baars2015a}. It has been established that the large-scales linearly superimpose themselves on the small-scales while also having a non-linear, amplitude and frequency modulation effect on the small-scales. 

\subsection{This is a Subsection Heading}
The principal investigator (PI) carried out a fundamental study characterizing this non-linear interaction through systematic perturbation of the large-scales. The model boundary layer chosen for this work is the the plane wall jet (PWJ). A PWJ is a two-dimensional jet that exits tangentially along a flat plate into either quiescent fluid or a fluid stream (co-flow) \citep{Launder1981a,Launder1983a} -- see schematic of Figure~\ref{fg:pwj_sch}. The PWJ considered was primarily into quiescent air. However, in the final year of performance a new PWJ facility with a co-flow was built and the resulting flow was studied. A PWJ has two shear layers that transition to turbulence through different mechanisms. On one hand, the outer free-shear layer (free jet portion),  transitions via an inviscid mechanism (Kelvin-Helmholtz instability), naturally leading to energetic large-scales. On the other hand, the inner shear layer (boundary layer portion) transitions through a viscous mechanism, which leads to finer scales of turbulence. These shear layers then interact as the flow develops downstream of the exit.

\begin{figure}[ht]
	\centering 
	\includegraphics[clip = true,width=0.9\textwidth]{pics/wallJetBl.png}%
	\caption{Schematic of a PWJ shear layer. Here, $x$ and $z$ denote streamwise and wall-normal directions respectively, $b$ the jet exit height, ${U}$ the mean streamwise velocity and $u$ the velocity fluctuation.}
	\label{fg:pwj_sch}
\end{figure}

The PWJ reaches a self-preserving state at distances greater than $x/b>40$ where $x$ is the stream-wise direction and $b$ the PWJ exit height (Figure ~\ref{fg:pwj_sch}). A schematic of the mean streamwise velocity profile $U$  is shown in Figure ~\ref{fg:pwj_sch}; the mean velocity has been non-dimensionalized using the maximum velocity $U_m$, which is the outer velocity scale. The outer length scale $\delta$ is the wall normal distance where the streamwise velocity is $U_m/2$, as shown in Figure~\ref{fg:pwj_sch}. Also shown is a schematic of the non-dimensionalized streamwise turbulence intensity profile. Seen clearly are two peaks, i.e. an outer larger peak corresponding to the energetic large outer scales and an inner smaller peak associated with the near-wall turbulence cycle as seen in canonical boundary layers (e.g., zero-pressure gradient, pipe flow). 
\subsubsection{This is a subsubsection heading}
The PWJ is chosen as the model flow field primarily for this reason -- a natural separation of scales exists as the turbulence arises from two distinct sources. Furthermore, the outer larger scales in a PWJ are extremely energetic, large-scales with inner-scaled energy density comparable to that seen in very high Reynolds numbers. The inherent configuration of the PWJ also allows for the control (or perturbation) of the outer larger scales independent of the inner cycle, particularly in a flow regime where the large-scales are extremely energetic.  On the other hand, the inner-cycle develops with energy density comparable to canonical boundary layers which allows the use of the PWJ to study the effect of large-scale perturbations. 

Studies have been carried out where the energetic large outer scales of the PWJ were modified~\cite{Katz1992a,Zhou1996a,Schober2000a}. \citet{Schober2000a} conducted experiments on a PWJ in which the outer shear layer structures were excited using an oscillating wire and also suppressed using a still wire. On the one hand, \citet{Schober2000a} and \citet{Zhou1996a} (\citet{Zhou1996a} used acoustic excitation) found that the excitation increased the coherence of the large-scales, which resulted in an increase in turbulence intensity of the streamwise velocity fluctuations. On the other hand, \citet{Schober2000a} showed that suppressing turbulence had the opposite effect. \citet{Schober2000a} also observed that there was a reduction in the mean (time-averaged) skin friction at the wall when the jet was excited; conversely there was an increase in skin friction when the turbulence was suppressed. \citet{Katz1992a} also reported a reduction in skin friction to various degrees based on the forcing frequency. These prior studies serve as a motivation for the present work. As part of the present work, it is sought to describe the internal mechanisms within a forced PWJ that lead to these changes in the near-wall region due to large-scale forcing. Lessons are also sought to be learnt that can aid in modelling and controlling complex wall-bounded flows a large.


\noindent \textbf{Relevance to the Air Force} - Apart from the basic physics benefit of understanding interactions within complex wall-bounded flows, the PWJ has other direct benefits to the Air Force. Due to the large-scale mixing present in the PWJ, the PWJ has been used as the prototypical model flow field to study mixing and reactions~\citep{Ahlman2007a,Pouransari2011a,Pouransari2013a}. Controlling the energetic, large-scales in the flow has direct relevance to controlling reaction and mixing rates in reacting systems. The nature of the interaction between the outer large-scales and the inner small-scales in a PWJ also has some similarities with the post-reattachment region of near-wall shear layers~\citep{Dejoan2005a} and jet impingements (seen in heating/cooling applications as well as in Vertical Take Off and Landing vehicles/rotorcraft in ground effect ~\citep{Agarwal1982a,Looney1984a,Jambunathan1992a,Rauleder2014a,Rauleder2014b}). PWJ type flows are also encountered both in the boundary layer development of multi-element airfoils and more prominently in film cooling of turbine blades and combustion chambers~\citep{Launder1983a,Wygnanski1992a}. In film cooling, the large-scales are the scales that are responsible for either enhanced or suppressed mixing of the cooling layer, with the surrounding combustion gases. The control of the large-scales in these flows offers the possibility of actively controlling the mixing between the layers. Thus the basic physics goals of this proposed work has relevance and significance to a wide variety of flow fields that are of interest to the Air Force, apart from the inherent insight obtained into complex wall-bounded flow in general.